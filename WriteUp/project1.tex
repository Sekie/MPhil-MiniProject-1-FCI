%%
%% Copyright 2007, 2008, 2009 Elsevier Ltd
%%
%% This file is part of the 'Elsarticle Bundle'.
%% ---------------------------------------------
%%
%% It may be distributed under the conditions of the LaTeX Project Public
%% License, either version 1.2 of this license or (at your option) any
%% later version.  The latest version of this license is in
%%    http://www.latex-project.org/lppl.txt
%% and version 1.2 or later is part of all distributions of LaTeX
%% version 1999/12/01 or later.
%%
%% This template has been modified by Philip Blakely for
%% local distribution to students on the MPhil for Scientific
%% Computing course run at the University of Cambridge.
%%

%% Template article for Elsevier's document class `elsarticle'
%% with numbered style bibliographic references
%% SP 2008/03/01
%%
%%
%%
%% $Id: elsarticle-template-num.tex 4 2009-10-24 08:22:58Z rishi $
%%
%%
\documentclass[final,3p,times,twocolumn]{elsarticle}

%% Use the option review to obtain double line spacing
%% \documentclass[preprint,review,12pt]{elsarticle}

%% Use the options 1p,twocolumn; 3p; 3p,twocolumn; 5p; or 5p,twocolumn
%% for a journal layout:
%% \documentclass[final,1p,times]{elsarticle}
%% \documentclass[final,1p,times,twocolumn]{elsarticle}
%% \documentclass[final,3p,times]{elsarticle}
%% \documentclass[final,3p,times,twocolumn]{elsarticle}
%% \documentclass[final,5p,times]{elsarticle}
%% \documentclass[final,5p,times,twocolumn]{elsarticle}

%% if you use PostScript figures in your article
%% use the graphics package for simple commands
%% \usepackage{graphics}
%% or use the graphicx package for more complicated commands
%% \usepackage{graphicx}
%% or use the epsfig package if you prefer to use the old commands
%% \usepackage{epsfig}

%% The amssymb package provides various useful mathematical symbols
\usepackage{amssymb}
%% The amsthm package provides extended theorem environments
%% \usepackage{amsthm}

%% The lineno packages adds line numbers. Start line numbering with
%% \begin{linenumbers}, end it with \end{linenumbers}. Or switch it on
%% for the whole article with \linenumbers after \end{frontmatter}.
%% \usepackage{lineno}

%% natbib.sty is loaded by default. However, natbib options can be
%% provided with \biboptions{...} command. Following options are
%% valid:

%%   round  -  round parentheses are used (default)
%%   square -  square brackets are used   [option]
%%   curly  -  curly braces are used      {option}
%%   angle  -  angle brackets are used    <option>
%%   semicolon  -  multiple citations separated by semi-colon
%%   colon  - same as semicolon, an earlier confusion
%%   comma  -  separated by comma
%%   numbers-  selects numerical citations
%%   super  -  numerical citations as superscripts
%%   sort   -  sorts multiple citations according to order in ref. list
%%   sort&compress   -  like sort, but also compresses numerical citations
%%   compress - compresses without sorting
%%
\biboptions{super}

% \biboptions{}

\usepackage{physics}
\usepackage{graphicx}
\usepackage[utf8]{inputenc}
\usepackage{amsmath}
\usepackage{mathtools}
%\usepackage{algorithm}
\usepackage[]{algorithm2e}

\newcommand{\ai}{\textit{ab-initio}}
\newcommand{\ssth}{\textsuperscript{th}}
\newcommand{\ham}{\hat{\mathcal{H}}}


\journal{MPhil in Scientific Computing}

\begin{document}

\begin{frontmatter}

%% Title, authors and addresses

%% use the tnoteref command within \title for footnotes;
%% use the tnotetext command for the associated footnote;
%% use the fnref command within \author or \address for footnotes;
%% use the fntext command for the associated footnote;
%% use the corref command within \author for corresponding author footnotes;
%% use the cortext command for the associated footnote;
%% use the ead command for the email address,
%% and the form \ead[url] for the home page:
%%
%% \title{Title\tnoteref{label1}}
%% \tnotetext[label1]{}
%% \author{Name\corref{cor1}\fnref{label2}}
%% \ead{email address}
%% \ead[url]{home page}
%% \fntext[label2]{}
%% \cortext[cor1]{}
%% \address{Address\fnref{label3}}
%% \fntext[label3]{}

\title{Full Configuration Interaction Calculations on BH and H$_2$O}

%% use optional labels to link authors explicitly to addresses:
%% \author[label1,label2]{<author name>}
%% \address[label1]{<address>}
%% \address[label2]{<address>}

\author{Henry Tran}

\address{Department of Chemistry, Lensfield Road, Cambridge, UK,
CB2 1EW}

\begin{abstract}
In this study, the method of full configuration interaction (FCI) to solve the electronic Schr\"{o}dinger Equation was implemented. The electronic Hamiltonian was diagonalized in a space of Slater determinants. The Davidson diagonalization algorithm was utilized and compared against a typical QR diagonalization routine, showing significant improvement in speed and no discernible loss in accuracy for LiH. This implementation of FCI was benchmarked against literature FCI calculations of the ground electronic states of BH and H$_2$O and a satisfactory agreement was found. The correlation energy was found to be large for these molecules, especially in the limit of dissociation.
\end{abstract}

\end{frontmatter}

%%
%% Start line numbering here if you want
%%
% \linenumbers

%% main text
\section{Introduction} \label{sec:intro}
The method of configuration interaction (CI) is ubiquitous in computational chemistry and has a rich and varied history.\cite{shavitt} CI builds upon the Hartree-Fock (HF) method,\cite{hartree,fock} which provides a cheap, approximate solution to the time-independent Schr\"{o}dinger Equation. Although HF is inexpensive, it approximates the true wavefunction as a single determinant, or configuration, and cannot account for correlation energy.\cite{szabo} Slater introduced the Slater determinant and showed how multiple configurations could be derived from the HF solution.\cite{slater} CI takes advantage of these configurations by introducing coupling between the different configurations, providing a solution that includes correlation energy. Condon further derived the rules to calculate the value of coupling between different configurations.\cite{condon} These configurations are formally a complete basis for the Hilbert Space in which the true solution lies. In practice, one uses a finite basis of either Slater determinants, or a linear combination of Slater determinants that are eigenstates of spin-angular momentum called configuration state functions (CSF). A solution using all possible $n$-electron determinants or CSFs that can be formed from the basis set is called full CI (FCI). FCI gives an exact solution to Schr\"{o}dinger's Equation in the functional space spanned by the basis set, but scales exponentially with system size, making practical calculations only possible for the smallest of molecules.\cite{szabo}

Early CI calculations used a Slater-type orbital (STO) basis set introduced by Slater.\cite{sto} The high cost of numerically integrating STOs made it very difficult to conduct \ai\ calculations on polyatomic systems.\cite{shavitt} Later, Boys introduced gaussian-type orbitals (GTO) as a basis to significantly simplify integrals.\cite{gto} One prominent calculation in the early days of CI includes the calculation of the three lowest lying electronic states of CH$_2$ in 1960.\cite{boys-1960} This calculation was the first to predict a bent ground state geometry for CH$_2$.

Much work has been dedicated to improving the efficiency and speed of CI calculations. While an FCI calculation provides a more complete solution than a calculation involving fewer configurations, FCI is not always feasible. However, new algorithms and computational tools have allowed for FCI calculations on systems of unprecedented sizes.\cite{bendazzoli-1995} Among these include a vectorized algorithm first introduced by Knowles and Handy for FCI in a determinant basis.\cite{handy-1984,handy-1988,olsen,evangelisti-1993}

Rather than going to the FCI limit, the configurations included in the CI expansion are typically truncated based on how many electrons are occupying virtual orbitals. For example, the inclusion of single excitations is called CIS. The inclusion of double excitations is called CISD, and so on.\cite{szabo}  

Though truncated CI allows for electronic structure calculations of much larger systems, they are not size-extensive. Only in the FCI limit is CI size-extensive, and it is for this reason coupled-cluster (CC) methods have taken over CI methods in popularity for mid-sized molecules.\cite{cc,bartlett-2007} Pople and Head-Gordon later formulated quadratic CI (QCI), which modified CI to be size-extensive.\cite{qci} QCI has been shown to be comparable to CC methods.\cite{werner-1992}

CI methods can be classified as single reference CI (SRCI) or multireference CI (MRCI). In SRCI, the CI expansion is based on one configuration, typically a HF solution. For MRCI, the expansion is based on a set of configuration references.\cite{hackmeyer-1969} Early choices for the reference configurations tended to be based on physical considerations\cite{shavitt-1974} or perturbative estimates for the most important configurations to the ground state.\cite{nesbet-1955} Later, more systematic choices were adopted. One choice is to take a set of ``active'' orbitals, typically valence orbitals, and then form configurations using the orbitals in the active space as references.\cite{cas} This method is known as complete active space (CAS). In some cases, CAS provides too many references to be computationally feasible. Another approach is to put an occupancy restriction on active space orbitals, in a method called restricted active space (RAS).\cite{ras} There is active work on further restrictions,\cite{shavitt} but they will not be mentioned here.

In terms of choosing CSFs, many methods exist including symmetric group methods,\cite{wigner} projection operator methods,\cite{lowdin} and unitary group methods.\cite{paldus-1974} These methods serve to construct a basis of eigenstates of total spin-angular momentum. However, no spin adaption will be done in this paper as the basis used in this study is the basis of Slater determinants. % Basis set? Orbitals?

The most expensive procedure in the CI algorithm is the diagonalization of the Hamiltonian matrix in the space of potentially millions of CSFs or determinants. However, in any CI calculation, one is typically only interested in a few lower lying solutions. Iterative methods such as the Lanczos algorithm are known to be effective for these situations.\cite{Lanczos} Davidson introduced a similar iterative algorithm that exploits the diagonal dominance of the Hamiltonian matrix to compute specific eigenvalues of the Hamiltonian without need to calculate lower eigenvalues first.\cite{davidson} Since then, a modified version of the Davidson algorithm was introduced by Liu to allow the algorithm to converge to multiple eigenvalues at once.\cite{liu}

This paper will discuss the application of the FCI method on the BH and H$_2$O molecules, which have been previously studied using FCI.\cite{cccbdb,handy-1983} The Hamiltonian matrix will be constructed in a space of Slater determinants and the Davidson diagonalization algorithm will be implemented. 

\section{Theory}
\subsection{Electronic Problem} \label{sec:elproblem}
The molecular Hamiltonian, ignoring higher order effects such as relativistic or fine-structure terms, is
\begin{equation}
\begin{gathered}
\hat{\mathcal H} = - \sum_I \dfrac{1}{2M_I} \nabla_I^2 -\dfrac{1}{2} \sum_i \nabla_i^2 - \sum_{iI} \dfrac{Z_I}{|\mathbf{R}_I - \mathbf{r}_i|} \\
+ \sum_{i < j} \dfrac{1}{|\mathbf{r}_i - \mathbf{r}_j|} + \sum_{I < J} \dfrac{Z_IZ_J}{|\mathbf{R}_I - \mathbf{R}_J|}
\end{gathered}
\end{equation}
The terms in order are: nuclear kinetic energy, electronic kinetic energy, electron-nuclear attraction, electron-electron repulsion, and nuclear-nuclear repulsion. Lowercase indexes label electrons, uppercase indexes label nuclei, $\mathbf{r}_i$ denotes the cartesian coordinates of the $i$\ssth\ electron, $\mathbf{R}_I$ denotes the cartesian coordinates of the $I$\ssth\ nucleus, and $Z_I$ is the atomic number of the $I$\ssth\ nucleus. All the electron coordinates will be denoted $\mathbf{r}$ and all the nuclear coordinates will be denoted $\mathbf{R}$. The last four terms are typically defined as the electronic Hamiltonian, $\ham_e$.
\begin{gather}
\hat{\mathcal H} = - \sum_I \dfrac{1}{2M_I} \nabla_I^2 + \ham_e \\
\begin{gathered}\label{eq:hame}
\ham_e = \hat T_e + \hat V_{eN} + \hat V_{ee} + \hat V_{NN} 
\end{gathered}
\end{gather}
where
\begin{subequations}
\begin{equation}
\hat T_e = -\dfrac{1}{2} \sum_i \nabla_i^2 
\end{equation}
\begin{equation}
\hat V_{eN} = - \sum_{iI} \dfrac{Z_I}{|\mathbf{R}_I - \mathbf{r}_i|}
\end{equation}
\begin{equation}
\hat V_{ee} = \sum_{i < j} \dfrac{1}{|\mathbf{r}_i - \mathbf{r}_j|}
\end{equation}
\begin{equation} \label{eq:vnn}
\hat V_{NN} = \sum_{I < J} \dfrac{Z_IZ_J}{|\mathbf{R}_I - \mathbf{R}_J|}
\end{equation}
\end{subequations}

The energy of a state is given by the eigenvalues of $\ham$. To greatly simplify the problem, the Born-Oppenheimer approximation\cite{bo} is made, which assumes that the electronic Hamiltonian can be solved at each fixed value of $\mathbf{R}$ to obtain electronic eigenstates parametrically dependent on $\mathbf{R}$.
\begin{equation}\label{eq:elproblem}
\ham_e \ket{\psi(\mathbf{r};\mathbf{R})} = E(\mathbf{R}) \ket{\psi(\mathbf{r};\mathbf{R})}
\end{equation}
where $E$ is the electronic energy and $\ket{\psi}$ are the electronic eigenstates. This paper focuses on solving Equation \eqref{eq:elproblem} for $E$ and $\ket{\psi}$.

\subsection{FCI Wavefunction} \label{sec:wf}
Equation \eqref{eq:elproblem} can be solved in theory by expanding $\ket{\psi}$ in a complete basis. The coefficients of the expansion are determined by minimizing the energy expectation functional, utilizing the variational theorem. This basis will be generated from a single reference determinant obtained from an HF calculation. 

The HF method\cite{hartree,fock,roothaan} minimizes the functional
\begin{equation}
E[\phi_1, \ldots, \phi_n] = \dfrac{\bra{\psi^\prime}\ham_e\ket{\psi^\prime}}{\bra{\psi^\prime}\ket{\psi^\prime}}
\end{equation}
where $n$ is the number of electrons in the system, $\ket{\phi_i}$ are spin orbitals constrained to be orthonormal, and 
\begin{equation} \label{eq:det}
\psi^\prime(\mathbf{r}) = \dfrac{1}{\sqrt{n!}}\det\begin{pmatrix} \phi_1(\mathbf{r}_1) & \cdots & \phi_n(\mathbf{r}_1) \\ \vdots & \ddots & \vdots \\
\phi_1(\mathbf{r}_n) & \cdots & \phi_n(\mathbf{r}_n) \end{pmatrix}
\end{equation}
$\ket{\psi^\prime}$ in this form is referred to as a Slater determinant\cite{slater} and will be denoted more succinctly as
\begin{equation}
\ket{\psi^\prime} = |\phi_1 \cdots \phi_n|
\end{equation}

From the HF method,\cite{roothaan} multiple spin orbitals can be obtained. The $n$ lowest energy spin orbitals are referred to as occupied orbitals. Denote the slater determinant formed by only occupied orbitals as $\ket{\psi_0}$. Higher energy spin orbitals are referred to as virtual orbitals, and put $m$ as the total number of virtual orbitals. Denote $\ket{\psi_{ij\ldots}^{ab\ldots}}$ as the Slater determinant formed from the occupied spin orbitals, but replacing occupied spin orbitals $i, j, \ldots$ with virtual spin orbitals $a, b, \ldots$ Note that there are $\binom{n+m}{n}$ possible Slater determinants, all orthonormal. The FCI method searches for solutions to Equation $\ref{eq:elproblem}$ within the determinant space spanned by all $\binom{n+m}{n}$ determinants. The $k$\ssth\ solution to Equation $\eqref{eq:elproblem}$ is written as an expansion in this determinant space
\begin{equation}\label{eq:fciwf}
\begin{gathered}
\ket{\psi^{(k)}} = c_0^{;k}\ket{\psi_0} + \sum_{\mathclap{\substack{i\in\text{occ} \\ a \in\text{vir}}}} c_i^{a;k}\ket{\psi_i^a} + \\
\sum_{\mathclap{\substack{i < j; i, j \in\text{occ} \\ a < b; a, b \in\text{vir}}}} c_{ij}^{ab;k}\ket{\psi_{ij}^{ab}} 
+ \cdots
\end{gathered}
\end{equation}
Truncation at the first term gives the HF wavefunction. Truncation at the second term gives the CIS wavefunction. Truncation at the third term gives the CISD wavefunction, and so on. Inclusion of all terms gives the FCI wavefunction.

\subsection{Slater-Condon Rules} \label{sec:condonrules}
In order to solve Equation \eqref{eq:elproblem} using Equation \eqref{eq:fciwf}, the energy expectation functional, $\bra{\psi}\hat{\mathcal H}\ket{\psi} / \bra{\psi}\ket{\psi}$, has to be minimized with respect to the coefficients in Equation \eqref{eq:fciwf}. This is equivalent to diagonalizing the electronic Hamiltonian matrix, $\mathcal H_e$, in the space of determinants used in the expansion. The matrix elements of $\mathcal H_e$ are given by
\begin{equation} \label{eq:hmat}
\mathcal H_{e;ij} = \bra{\psi_i}\ham_e\ket{\psi_j}
\end{equation}
where $\ket{\psi_i}$ now simply denote different slater determinants within the determinant space.

These matrix elements can be reduced to simpler integrals as summarized by the Slater-Condon rules.\cite{condon} First, $\ham_e$ is expressed as sum of one, two, and zero electron operators.
\begin{equation}
\ham_e = \sum_{i=1}^n \hat h_i + \sum_{i < j}^n \hat g_{ij} + \hat V_{NN}
\end{equation}
where
\begin{subequations}
\begin{equation} \label{eq:h}
\hat h_i = -\dfrac{1}{2} \nabla_i + \sum_I \dfrac{Z_I}{|\mathbf{R}_I - \mathbf{r}_i|}
\end{equation}
\begin{equation} \label{eq:g}
\hat g_{ij} = \dfrac{1}{|\mathbf{r}_i - \mathbf{r}_j|}
\end{equation}
\end{subequations}
and $\hat V_{NN}$ is defined in Equation \eqref{eq:vnn}.

The Slater-Condon rules are now presented. Consider two slater determinants $\ket{\psi_1}$ and $\ket{\psi_2}$ with maximum matching orbitals. This means that spin orbitals shared by both determinants are ordered so that they are in the same position within both determinants, with the caveat that each transpose changes the sign of the determinant. In the following equations, $i$ and $j$ indexes shared spin orbitals. The index on the operators are dropped when the operator acts on a dummy integration variable. The notation $(kl|pq)$ is also defined as
\begin{equation}
(kl|pq) = \int \phi_k^*(\mathbf{r})\phi_p^*(\mathbf{r}^\prime) \frac{1}{|\mathbf{r}-\mathbf{r}^\prime|} \phi_l(\mathbf{r}) \phi_q(\mathbf{r}^\prime) \mathrm{d}\mathbf{r}\mathrm{d}\mathbf{r}^\prime 
\end{equation}
The Slater-Condon rules are:
\begin{enumerate}
\begin{subequations}
\item If $\ket{\psi_1}$ and $\ket{\psi_2}$ are identical, then
\begin{equation} \label{eq:sc0}
\begin{gathered}
\bra{\psi_1} \ham_e \ket{\psi_2} = \sum_{i=1}^n \bra{\phi_i} \hat h \ket{\phi_i} \\ + \sum_{j > i}^n \left[ (jj|ii) - (ji|ji) \right] + V_{NN}
\end{gathered}
\end{equation}

\item If $\ket{\psi_1}$ and $\ket{\psi_2}$ differ by one spin orbital,
\begin{align*}
\ket{\psi_1} = |\cdots\phi_k\cdots| \\ 
\ket{\psi_2} = |\cdots\phi_p\cdots| 
\end{align*}
then
\begin{equation} \label{eq:sc1}
\begin{split}
\bra{\psi_1} \ham_e \ket{\psi_2} = & \bra{\phi_k} \hat h \ket{\phi_p} \\ & + \sum_{i=1}^n \left[ (kp|ii) - (ki|ip) \right]
\end{split}
\end{equation}

\item If $\ket{\psi_1}$ and $\ket{\psi_2}$ differ by two spin orbitals,
\begin{align*}
\ket{\psi_1} = |\cdots\phi_k\phi_l\cdots| \\ 
\ket{\psi_2} = |\cdots\phi_p\phi_q\cdots| 
\end{align*}
then
\begin{equation} \label{eq:sc2}
\bra{\psi_1} \ham_e \ket{\psi_2} = (kp|lq) - (kq|lp)
\end{equation}

\item If $\ket{\psi_1}$ and $\ket{\psi_2}$ differ by three or more spin orbitals, then
\begin{equation} \label{eq:sc3}
\bra{\psi_1} \ham_e \ket{\psi_2} = 0
\end{equation}
\end{subequations}
\end{enumerate}

It is clear that setting up an FCI calculation requires the evaluation of many integrals. To calculate these integrals, the spin orbitals are approximated by a sum of GTOs,\cite{gto} the number and form of which depending on the type of basis desired. STOs are rarely used because STOs must be numerically integrated, whereas GTOs have analytic integrals.\cite{szabo} The details of the basis used in this study will be discussed in Section \ref{sec:results}.

\section{Procedure}

In this study, the FCI Hamiltonian matrices for one geometry of BH and three different geometries of H$_2$O were generated using the Slater-Condon rules presented in Equations \eqref{eq:sc0} to \eqref{eq:sc3}. The electronic structure package, Q-Chem,\cite{qchem} was used to conduct an initial HF calculation and then to obtain the integrals in Equations \eqref{eq:sc0} to \eqref{eq:sc3}. The Hamiltonian matrices were then diagonalized using the Davidson diagonalization algorithm\cite{davidson,liu} to find the lowest eigenvalue. This algorithm will be discussed in Section \ref{sec:davidson}. The FCI results will be compared to the initial HF calculation done in Q-Chem and literature results\cite{cccbdb,handy-1983} in Section \ref{sec:results}.

The Eigen library was used for matrix and vector algebra.\cite{eigen} OpenMP was used to parallelize the generation of the matrix and Eigen's implementation of OpenMP was used to parallelize operations used in the diagonalization algorithm.\cite{openmp} The matrix was stored in sparse matrix form with single precision entries. All matrix elements less than 10$^{-15}$ $E_h$ were not stored in the matrix to reduce memory costs. %Q-Chem was carried out on a single 12-core computer. 
The matrix generation and diagonalization was carried out on a single computer with a quad-core processor and 64GB of RAM.

%\subsection{Computational Details} \label{sec:integrals}

\subsection{Complete (QR) Diagonalization} \label{sec:qr}
All complete diagonalizations were conducted using the Eigen library.\cite{eigen} Eigen uses a QR algorithm\cite{qr} for complete diagonalization, and this algorithm will be summarized in this section.

At each step in the QR algorithm, the current matrix is decomposed into its QR decomposition, $H_k = Q_kR_k$ where $Q_k$ is orthogonal and $R_k$ is upper triangular. The next matrix is generated as $H_{k+1} = R_kQ_k$. This converges to a triangular matrix, and the eigenvalues are then the diagonal elements of the triangular matrix. A pseudo-code is presented in Algorithm \ref{alg:qr}. The cost of a QR diagonalization is $\mathcal O(N^3)$ where $N$ is the dimension of the matrix.

\begin{algorithm}
 Read in $N \times N$ Hamiltonian matrix\;
 
 Choose a set of orthonormal basis vectors $\{\mathbf{e}^i\}$ with $i = 1,  \ldots, N$\;
 
 \While{$H$ is not triangular}{
 Form $R$ with elements $R_{ij} = \sum_I^N e^i_I H_{Ij}$ for $j \geq i$ and $R_{ij} = 0$ otherwise\;
 
 Form $Q = (\mathbf{e}^1, \ldots, \mathbf{e}^N)$\;
 
 $H \leftarrow RQ$
 }
 The eigenvalues are $H_{ii}$ for $i = 1, \ldots, N$\;
\caption{The QR algorithm to find all eigenvalues of an $N 
\times N$ matrix, $H$.} 
\label{alg:qr}
\end{algorithm}

\subsection{Davidson Diagonalization} \label{sec:davidson}
The size of the determinant space can be over one million determinants. The cost of diagonalizing the full Hamiltonian matrix in Equation \eqref{eq:hmat} is usually prohibitively large. %In this study, the package Eigen for C++ was utilized to diagonalize the full Hamiltonian.\cite{eigen} Eigen uses a QR iterative algorithm that has a cubic cost and cubic convergence.\cite{qr}

Because of the nature of our problem, a full diagonalization is not necessary. Typically, only the lowest eigenvalues of Equation \eqref{eq:elproblem} are desired. Davidson introduced an iterative eigenvalue algorithm that was capable of converging to a chosen, the lowest for example, eigenvalue in a significantly shorter amount of time.\cite{davidson} Liu later adapted the algorithm to converge to multiple eigenvalues at once.\cite{liu} Liu's formulation is presented in Algorithm \ref{alg:davidson} and this is the formulation that has been implemented in this work.

\begin{algorithm}
 Read in $N \times N$ Hamiltonian matrix, $H$\;
 
 Choose a set of orthonormal trial vectors $B = \{\mathbf{b}^i\}$ with $i = 1,  \ldots, L$ where $L \geq M$\;
 
 $EVFound \leftarrow \varnothing $\;
 
 \While{$|EVFound| < M$}{
  Set $G_{ij} \leftarrow \mathbf{b}^{iT} H \mathbf{b}^j$\;
  Find eigenvalues $\lambda_k$ and eigenvectors $\mathbf{a}^k$ of $G$ and keep the $M$ lowest eigenvalues\;

  \For{k = 1, \ldots, M}
  {
  	$\mathbf{d}^k \leftarrow \sum_i^L a_i^k(H - \lambda^k) \mathbf{b}^i$\;
  	\If{$\mathbf{d}^k \cdot \mathbf{d}^k < 10^{-6}$}
  	{
  		$EVFound \leftarrow EVFound \cup \{(\lambda^k, \mathbf{a}^k)\}$\;
  	}
  }
  
  \If{$|EVFound| < M$}
  {
  	$EVFound = \varnothing$\;
  }
    
  \For{k = 1, \ldots, M}
  {
  	\For{I = 1, \ldots, N}
  	{
  		$f_I^k \leftarrow (\lambda^k - H_{II})^{-1}d_I^k$\;
  	}
  	Form residual vector $\mathbf{f}^k = (f_1^k, \ldots, f_N^k)$\;
  }
  
  
  Orthonormalize $\mathbf{f}^1$ with $B$\;
  $B \leftarrow B \cup \{\mathbf{f}^1\}$\;
  $m \leftarrow 1$\;
  
  \For{k = 2, \ldots, M}
  {
  	Orthogonalize $\mathbf{f}^k$ with $B$\;
  	\If{$\mathbf{f}^k \cdot \mathbf{f}^k > 10^{-3}$}
  	{
  		Normalize $\mathbf{f}^k$\;
  		$B \leftarrow B \cup \{\mathbf{f}^k\}$\;
  		$m \leftarrow m + 1$\;
  		
  	}
  }
  Reorthonormalize $B$\;
  $L \leftarrow L + m$\;
 }
 $EVFound$ contains the desired eigenpairs\;

\caption{The Davidson algorithm to find the $M$ lowest eigenvalues of a matrix.} 
\label{alg:davidson}
\end{algorithm}

\section{Results} \label{sec:results}

%Q-Chem was used to compute the integrals in Equations \eqref{eq:sc0} to \eqref{eq:sc3}. These integrals were used to form the FCI Hamiltonian matrix for one geometry of BH and three geometries of H$_2$O. The geometries are listed in Table \ref{tab:geo}
Table \ref{tab:geo} contains a summary of the geometries used in the calculations of each molecule studied.

\begin{table}
\centering
\begin{tabular}{l|c}
\hline\hline
& Bond Length (\AA) \\ \hline
BH & 1.2449 \\ \hline
H$_2$O (1) & 0.9755\\
H$_2$O (2) & 1.4633\\
H$_2$O (3) & 1.9510\\ \hline\hline
\end{tabular}
\caption{The geometries of BH and H$_2$O used in the calculations. Bond length refers to the B---H bond for BH and both O---H bonds for H$_2$O. The HOH bond angle is set at 110.565$^\circ$. The geometry for BH and the first geometry for H$_2$O correspond to equilibrium geometries.}
\label{tab:geo}
\end{table}


\subsection{Comparison of QR and Davidson Diagonalization}
The memory required to store and diagonalize an FCI matrix is prohibitively large, especially for the systems under consideration in the following sections. Therefore, the Davidson algorithm is called upon to obtain the eigenvalues of these matrices. In order to test the validity of the Davidson method, an analysis on a smaller system, LiH, was conducted. The FCI Hamiltonian for LiH was generated in a 6-31G* basis.\cite{bhbasis} The bond length was set to 1.6531 \AA. The calculation involved 4 electrons in a space of 17 orbitals. The determinant space was spanned by 18496 determinants. Two different methods of diagonalization were tested. The first method utilized the QR algorithm detailed in Section \ref{sec:qr} as implemented in Eigen\cite{eigen} and the second method utilized the Davidson algorithm in Section \ref{sec:davidson}. The results are compared to an FCI result from the CCCBDB.\cite{cccbdb} The wall time taken to complete each calculation is also recorded. The results are listed in Table \ref{tab:lihresults}.

\begin{table}
\centering
\begin{tabular}{l|ccc} \hline\hline
& Lit\cite{cccbdb} & Davidson & QR \\ \hline
FCI Energy ($E_h$) & 8.0035 & 8.0035 & 8.0035 \\
Time (s) & --- & 5 & 2961 \\ \hline\hline
\end{tabular}
\caption{FCI calculations of LiH using different diagonalization techniques. The energy eigenvalues and wall time of each calculation are listed.}
\label{tab:lihresults}
\end{table}

It is seen that the eigenvalues obtained through direct (QR) diagonalization agree numerically with those obtained through iterative (Davidson) diagonalization. This lends credibility that the Davidson algorithm yields the correct eigenvalues. Both methods yield results that agree with literature values. However, there is a substantial difference in time taken to complete these calculations. Both time and computer memory become prohibitive factors as the space of determinant grows and for that reason, the following sections will utilize only the Davidson diagonalization method to obtain the eigenvalues of the FCI matrix.


\begin{table}
\centering
\begin{tabular}{l|c|ccc} \hline\hline
& FCI (Lit) & FCI & HF & $E_{\text{corr}}$ \\ \hline
BH & -25.2206 & -25.2206 & -25.1181 & -0.1025 \\ \hline
H$_2$O (1) & -76.1579 & -76.1579 & -76.0098 & -0.1481 \\ 
H$_2$O (2) & -76.0145 & -76.0145 & -75.8035 & -0.2110 \\
H$_2$O (3) & -75.9052 & -75.9052 & -75.5952 & -0.3100 \\\hline\hline
\end{tabular}
\caption{Calculated ground state energies of BH at the geometry specified in Table \ref{tab:geo}. All units are in $E_h$. The calculation for BH is done using a 6-31G* basis\cite{bhbasis} and for H$_2$O is done using a DZ basis.\cite{dunning} Literature values for BH are taken from the CCCBDB\cite{cccbdb} and for H$_2$O are taken from a previous study by Harrison and Handy.\cite{handy-1983}}
\label{tab:results}
\end{table}
\subsection{BH} \label{sec:bhresults}

For BH, the bond length was set to the equilibrium bond length of 1.2449 \AA. A 6-31G* basis\cite{bhbasis} was used for the HF calculation and the calculation of the integrals. The calculation involved 6 electrons and 17 orbitals. The determinant space was spanned by 462400 determinants.

The results of the FCI and HF calculation of BH are listed in Table \ref{tab:results}. The correlation energy is calculated as $E_{\text{FCI}} = E_{\text{FCI}} - E_{\text{HF}}$. Also listed is the FCI energy calculated from the CCCBDB using the same basis.\cite{cccbdb} It is noted that although Harrison and Handy have conducted an FCI analysis of BH using a DZ* basis,\cite{handy-1983} the calculations from that study study disagree with cross-tested calculations in Q-Chem. The analysis from Harrison and Handy either contains an error or the basis set is not fully documented, so for the purpose of this study the data from the CCCBDB will be used instead. The FCI energy from this study (-25.2206 $E_h$) agrees numerically to the previous study, which lends credibility to the calculations in this study. %The disagreement between the two calculations is likely a result of numerical instability in the implementation of the Davidson diagonalization used in this paper, which involves repeated orthogonalization of a set of vectors.

%The FCI energy from this study (-25.2295 $E_h$) and from the previous study (-25.2276 $E_h$) differ by 0.0019 $E_h$, which is very good and lends credibility to the calculations in this study.

The HF calculation returned an energy of -25.1181 $E_h$, which means that the HF calculation neglects a correlation energy of -0.1025 $E_h$. This is a significant amount of correlation energy, implying that post-HF methods should be applied to study BH.


\subsection{H$_2$O} \label{sec:h2oresults}

For H$_2$O, the HOH angle was set to 110.565$^\circ$ and O---H bond lengths were varied between three different values listed in Table \ref{tab:geo}. The first geometry corresponds to the equilibrium geometry. The second and third geometries correspond to points along the symmetric stretch. To calculate the HF energy and integrals at each geometry, a DZ basis was used.\cite{dunning} These geometries and the basis set match those used in a previous work.\cite{handy-1983} The calculation involved 10 electrons and 14 orbitals for each geometry. In each calculation, the determinant space was spanned by 4008004 determinants.

The results of the FCI and HF calculation of H$_2$O are listed in Table \ref{tab:results} for all three geometries of H$_2$O tested in this study. Also listed is the FCI ground state energy determined by Harrison and Handy using the same geometry and basis.\cite{handy-1983} Correlation energy is calculated as $E_{\text{corr}} = E_{\text{FCI}} - E_{\text{HF}}$. The FCI ground state energies of this study all agree with the previous study to within 10$^{-3}$ $E_h$. %0.0017 $E_h$ at worst.

Of particular interest is the progression of correlation energy as H$_2$O moves away from equilibrium geometry. The correlation energy at equilibrium geometry is -0.149 $E_h$, and this grows to -0.210 $E_h$ and -0.309 $E_h$ at 1.5 and 2 times the equilibrium bond lengths respectively. This illustrates that the HF prediction of ground state energy decays in quality as the molecule moves away from equilibrium, and hence potential energy surfaces computed using HF are prone to large errors. Higher levels of theory or the unrestricted formulation of HF (UHF) should be applied to cases such as dissociation, when the molecule moves far away from its equilibrium geometry.

It is also worth noting that the FCI calculation for Geometry 3 in the previous study cost four hours.\cite{handy-1983} That study was conducted in 1983. The FCI calculation in this study cost 39 minutes. Although FCI is expensive, it is still practical for many molecules and is becoming more effective as computing power develops.


\section{Conclusions and Outlook}
\label{sect:Concl}
The electronic energies of the ground states of BH and H$_2$O were computed at the FCI and HF levels of theory. The electronic FCI Hamiltonians for these molecules were diagonalized in a space of Slater determinants using integrals calculated from Q-Chem. The Davidson diagonalization algorithm was implemented for the diagonalization. An implementation utilizing Davidson diagonalization was compared to an implementation using QR diagonalization for LiH. Both implementations gave the same results, but the Davidson diagonalization was significantly faster. The FCI results agree well with a previous study\cite{handy-1983} and it is seen that correlation energy grows rapidly as the molecule moves away from its equilibrium geometry.

Although computationally costly, FCI remains feasible for a wide range of molecules and is effective in determining the most accurate electronic energies possible for these molecules. The Davidson diagonalization algorithm implemented in this study proved to be an efficient method to obtain the lower eigenvalues of the Hamiltonian. In future studies, a vectorized algorithm will be implemented to circumvent the need to store the Hamiltonian matrix and take advantage of parallelized computer architectures.\cite{handy-1983} It will also be worthwhile to compare the FCI results to results from truncated CI, such as CISD, and to results using CC methods to judge the relative accuracy of other methods. In this study, the correlation energy was determined, but it will be more insightful to test how much of the correlation energy is recovered with the inclusion of singles, doubles, and so on.

\section*{Acknowledgements}
I thank my supervisor, Dr.\ Alex Thom, and the rest of my group for their guidance and patience. I would also like to thank the Sir Winston Churchill Foundation of the USA for funding.

%% The Appendices part is started with the command \appendix;
%% appendix sections are then done as normal sections
%\appendix

%\section{Derivation of the Slater-Condon Rules}
%\label{app:slatercondon}
%For the derivation, the slater determinant will be written in the form
%\begin{equation}
%|\phi_1 \ldots \phi_n| = \dfrac{1}{\sqrt{n!}}\sum_{\sigma\in S_n} \text{sign}(\sigma) \phi_{\sigma(1)}(\mathbf{r}_1) \cdots \phi_{\sigma(n)}(\mathbf{r}_n)
%\end{equation}
%where $S_n$ denotes the permutation group of $n$ elements and sign($\sigma$) is +1 for even permutations, -1 for odd permutations. The proofs are presented below, proved in the same order as presented in Section \ref{sec:condonrules} and using the same notation.
%
%\begin{enumerate}
%	\item For the one element operator matrix element, 
%	\begin{gather}
%	\bra{\psi_1}\sum_i\hat h_i\ket{\psi_1}\\ 
%	\begin{gathered}
%	= \frac{1}{n!}\sum_{\sigma\sigma^\prime \in S_n} \sum_{i=1}^n \text{sign}(\sigma)\text{sign}(\sigma^\prime)
%	 \\ \times \bra{\phi_{\sigma(1)} \cdots \phi_{\sigma(n)}}\hat h_i \ket{\phi_{\sigma^\prime(1)} \cdots \phi_{\sigma^\prime(n)}} 
%	\end{gathered} \\
%	 \begin{gathered} = \frac{1}{n!}\sum_{\sigma\sigma^\prime \in S_n} \sum_{i=1}^n \text{sign}(\sigma)\text{sign}(\sigma^\prime) \\
%	 \times \bra{\phi_{\sigma(i)}} \hat h_i \ket{\phi_{\sigma^\prime(i)}} \prod_{k\neq i}^n \bra{\phi_{\sigma(k)}}\ket{\phi_{\sigma^\prime(k)}}
%	 \end{gathered}
%	\end{gather}
%	Because the spin orbitals are orthonormal, the product is zero when $\sigma \neq \sigma^\prime$ and is unity otherwise.
%	\begin{gather}
%	= \frac{1}{n!} \sum_{\sigma \in S_n} \text{sign}(\sigma)^2\sum_{i=1}^n \bra{\phi_{\sigma(i)}} \hat h_i \ket{\phi_{\sigma(i)}}
%	\end{gather}
%	The sign term squares to one always. For any $\sigma$, the second sum will be the same and there are $n!$ possible $\sigma$.
%	\begin{gather}
%	= \sum_{i=1}^n \bra{\phi_i}\hat h_i \ket{\phi_i}
%	\end{gather}
%	
%	For the two electron operator matrix element,
%	\begin{gather}
%	\bra{\psi_1}\sum_{i<j}^n\hat g_{ij}\ket{\psi_1} \\
%	\begin{gathered}
%	= \frac{1}{n!}\sum_{\sigma\sigma^\prime\in S_n} \sum_{i < j}^n \text{sign}(\sigma)\text{sign}(\sigma^\prime) \\
%	\times \bra{\phi_{\sigma(i)}\phi_{\sigma(j)}}\hat g_{ij} \ket{\phi_{\sigma^\prime(i)}\phi_{\sigma^\prime(j)}} \\
%	\times \prod_{k \neq i,j}^n \bra{\phi_{\sigma(k)}}\ket{\phi_{\sigma^\prime(k)}}
%	\end{gathered}
%	\end{gather}
%	The third line is zero unless $\sigma$ and $\sigma^\prime$ agree for all $k$ not equal to $i$ or $j$. In this case, $\sigma$ and $\sigma^\prime$ may agree for $i$ and $j$, or switch so that $\sigma(i) = \sigma^\prime(j)$. In the latter case, the relative sign of $\sigma^\prime$ is opposite as well.
%	\begin{gather}
%	\begin{gathered}
%	= \frac{1}{n!} \sum_{\sigma \in S_n} \text{sign}(\sigma) \\ \times \sum_{i < j}^n \left[ \bra{\phi_{\sigma(i)}\phi_{\sigma(j)}}\hat g_{ij} \ket{\phi_{\sigma(i)}\phi_{\sigma(j)}} - \right.\\ \left. \bra{\phi_{\sigma(i)}\phi_{\sigma(j)}}\hat g_{ij} \ket{\phi_{\sigma(j)}\phi_{\sigma(i)}} \right]
%	\end{gathered} \\
%	\begin{gathered}
%	= \frac{n!}{n!} \sum_{i < j}^n \left[ \bra{\phi_i\phi_j}\hat g_{ij} \ket{\phi_i\phi_j} - \right.\\ \left. \bra{\phi_i\phi_j}\hat g_{ij} \ket{\phi_j\phi_i} \right]
%	\end{gathered} \\
%	= \sum_{i < j}^n \left[ (ii|jj) - (ij|ij) \right]
%	\end{gather}
%	Finally, the zero electron operator gives
%	\begin{equation}
%	\bra{\psi_1}\hat V_{NN} \ket{\psi_1} = V_{NN} \bra{\psi_1}\ket{\psi_1} = V_{NN}
%	\end{equation}
%	
%	\item For the one electron operator matrix element,
%	\begin{gather}
%	\begin{gathered}
%	\bra{\psi_1}\sum_{i=1}^n \hat h_i \ket{\psi_2} = \\ \frac{1}{n!} \sum_{\sigma, \sigma^\prime \in S_n}\sum_{i=1}^n \text{sign}(\sigma)\text{sign}(\sigma^\prime) \\
%	\times \bra{\cdots\phi_{\sigma(k)}\cdots} \hat h_i \ket{\cdots\phi_{\sigma^\prime(p)}\cdots}
%	\end{gathered}
%	\end{gather}
%	The position of the unique orbital will be denoted $p$, but if $p$ appears in the bra, it implies $k$.
%	\begin{gather}
%	\begin{gathered}
%	\bra{\psi_1}\sum_{i=1}^n \hat h_i \ket{\psi_2} = \\ \frac{1}{n!} \sum_{\sigma, \sigma^\prime \in S_n}\sum_{i=1}^n \text{sign}(\sigma)\text{sign}(\sigma^\prime) \\
%	\times \bra{\phi_{\sigma(i)}}\hat h_i \ket{\phi_{\sigma^\prime(i)}} \prod_{j\neq i}^n \bra{\phi_{\sigma(j)}}\ket{\phi_{\sigma^\prime(j)}}
%	%\times \bra{\phi_{\sigma(k)}}\ket{\phi_{\sigma^\prime(p)}}
%	\end{gathered}
%	\end{gather}
%	Again, $\sigma$ and $\sigma^\prime$ must agree to be nonzero.
%	\begin{gather} \label{eq:hexp}
%	\begin{gathered}
%	 \frac{1}{n!} \sum_{\sigma \in S_n}\sum_{i=1}^n \text{sign}(\sigma)^2 \\
%	\times \bra{\phi_{\sigma(i)}}\hat h_i \ket{\phi_{\sigma(i)}} \prod_{j\neq i}^n \bra{\phi_{\sigma(j)}}\ket{\phi_{\sigma(j)}}
%	\end{gathered}
%	\end{gather}
%	In this case, another possibility is that $\sigma(j) = p$. Since $\phi_p$ is not an orbital in $\ket{\psi_1}$, the term above is zero whenever $\sigma(j) = p$ for all $j \neq i$. This forces $\sigma(i) = p$ and correspondingly, $\sigma(i) = k$ in the bra. The repeated product is then unity. If $i$ is fixed to $k$ or $p$, then there are $(n-1)!$ ways to permute the remaining elements.
%	\begin{gather}
%	 \frac{1}{n!}(n-1)!\sum_{i=1}^n \bra{\phi_{k}}\hat h_i \ket{\phi_{p}} \\
%	 \frac{1}{n!}n(n-1)!\bra{\phi_{k}}\hat h \ket{\phi_{p}} \\
%	 \bra{\phi_k}\hat h \ket{\phi_p}
%	\end{gather}
%	
%	For the matrix element of the two electron operator,
%	\begin{gather}
%	\bra{\psi_1}\frac{1}{2}\sum_{i=1}^n\sum_{j\neq i}^n \hat g_{ij}\ket{\psi_2} \\
%	\begin{gathered}
%	= \frac{1}{2}\frac{1}{n!}\sum_{\sigma\sigma^\prime\in S_n} \sum_{i \neq j}^n \text{sign}(\sigma)\text{sign}(\sigma^\prime) \\
%	\times \bra{\phi_{\sigma(i)}\phi_{\sigma(j)}}\hat g_{ij} \ket{\phi_{\sigma^\prime(i)}\phi_{\sigma^\prime(j)}} \\
%	\times \prod_{k \neq i,j}^n \bra{\phi_{\sigma(k)}}\ket{\phi_{\sigma^\prime(k)}}
%	\end{gathered}
%	\end{gather}
%	Again, the third term is zero whenever $\sigma$ and $\sigma^\prime$ disagree on $k \neq i,j$, so among the nonzero permutations are the ones which fix $i$ and $j$, and which switch $i$ and $j$.
%	\begin{gather} \label{eq:gexp}
%	\begin{gathered}
%	= \frac{1}{2}\frac{1}{n!} \sum_{\sigma \in S_n} \sum_{i \neq j}^n  \text{sign}(\sigma) \\ \times \left[ \bra{\phi_{\sigma(i)}\phi_{\sigma(j)}}\hat g_{ij} \ket{\phi_{\sigma(i)}\phi_{\sigma(j)}} - \right.\\ \left. \bra{\phi_{\sigma(i)}\phi_{\sigma(j)}}\hat g_{ij} \ket{\phi_{\sigma(j)}\phi_{\sigma(i)}} \right] \\
%	\times \prod_{k \neq i,j}^n \bra{\phi_{\sigma(k)}}\ket{\phi_{\sigma^\prime(k)}}
%	\end{gathered}
%	\end{gather}
%	The repeated product over $k$ is zero whenever $\sigma(k) = p$ (or $k$ in $\ket{\psi_1}$) because that orbital is not contained in $\ket{\psi_1}$. Hence, it must be that the only nonzero permutations are ones such that $\sigma(i) = p$ or $\sigma(j) = p$.
%	\begin{gather}
%	\begin{gathered}
%	= \frac{1}{2}\frac{1}{n!} \sum_{i \neq j}^n \sum_{\sigma\in S_n} \text{sign}(\sigma) \times \\ \left[ \bra{\phi_{k}\phi_{\sigma(j)}}\hat g_{ij} \ket{\phi_{p}\phi_{\sigma(j)}} - \right.
%	\\ \left. \bra{\phi_{k}\phi_{\sigma(j)}}\hat g_{ij} \ket{\phi_{\sigma(j)}\phi_{p}} \right] \\
%	+ \left[ \bra{\phi_{\sigma(i)}\phi_{k}}\hat g_{ij} \ket{\phi_{\sigma(i)}\phi_{p}} - \right.\\ \left. \bra{\phi_{k}\phi_{\sigma(i)}}\hat g_{ij} \ket{\phi_{p}\phi_{\sigma(i)}} \right] 
%	\end{gathered}
%	\end{gather}
%	If $i = j$, then the first and fourth term will cancel out, as well as the second and third term, so we can remove the restriction on the sum. Moreover, the dummy variable of integration can be exchanged on the first and third term to show that these terms are the same, when summed over $i$ and $j$. The same is true for the second and fourth terms.
%	\begin{gather}
%	\begin{gathered}
%	= \frac{1}{2}\frac{1}{n!} \sum_{j = 1}^n \sum_{\sigma\in S_n} \text{sign}(\sigma) \times \\ \left[ \bra{\phi_{k}\phi_{\sigma(j)}}\hat g_{ij} \ket{\phi_{p}\phi_{\sigma(j)}} - \right.
%	\\ \left. \bra{\phi_{k}\phi_{\sigma(j)}}\hat g_{ij} \ket{\phi_{\sigma(j)}\phi_{p}} \right] \\
%	+ \sum_{i = 1}^n \left[ \bra{\phi_{\sigma(i)}\phi_{k}}\hat g_{ij} \ket{\phi_{\sigma(i)}\phi_{p}} - \right.\\ \left. \bra{\phi_{k}\phi_{\sigma(j)}}\hat g_{ij} \ket{\phi_{p}\phi_{\sigma(i)}} \right] 
%	\end{gathered} \\
%	\begin{gathered}
%	= \frac{1}{2}\frac{1}{n!} 2 \sum_{i = 1}^n \sum_{\sigma\in S_n} \text{sign}(\sigma) \times \\ \left[ \bra{\phi_{k}\phi_{\sigma(i)}}\hat g_{ij} \ket{\phi_{p}\phi_{\sigma(i)}} - \right.
%	\\ \left. \bra{\phi_{k}\phi_{\sigma(i)}}\hat g_{ij} \ket{\phi_{\sigma(i)}\phi_{p}} \right]
%	\end{gathered} \\
%	= \frac{2\cdot n!}{2\cdot n!} \sum_{i = 1}^n \left[(kp|ii) - (ki|ip)\right] \\
%	= \sum_{i = 1}^n \left[(kp|ii) - (ki|ip)\right]
%	\end{gather}
%	
%	Finally, the matrix element of the zero electron operator is
%	\begin{gather}
%	\bra{\psi_1} \hat V_{NN} \ket{\psi_2} = V_{NN}\bra{\psi_1}\ket{\psi_2} = 0
%	\end{gather}
%	
%	\item For the one electron operator, the matrix element reduces the same way as was done in previous cases to Equation \eqref{eq:hexp}.
%%	\begin{gather}
%%	\begin{gathered}
%%	\bra{\psi_1}\sum_{i=1}^n \hat h_i \ket{\psi_2} = \\ \frac{1}{n!} \sum_{\sigma, \sigma^\prime \in S_n}\sum_{i=1}^n \text{sign}(\sigma)\text{sign}(\sigma^\prime) \\
%%	\times \bra{\cdots\phi_{\sigma(k)}\phi_{\sigma(l)}\cdots} \hat h_i \ket{\cdots\phi_{\sigma^\prime(p)}\phi_{\sigma(q)}\cdots}
%%	\end{gathered} \\
%%		\begin{gathered}
%%	 =\frac{1}{n!} \sum_{\sigma \in S_n}\sum_{i=1}^n \bra{\phi_{\sigma(i)}}\hat h_i \ket{\phi_{\sigma(i)}} \\ \times \prod_{j\neq i}^n \bra{\phi_{\sigma(j)}}\ket{\phi_{\sigma(j)}}
%%	\end{gathered}
%%	\end{gather}
%	In this case, if $\sigma(j)$ is $p$ or $q$, then the product over $j$ will be zero. However, since $j$ loops over all but one element, it must be that $\sigma(j) = p$ or $\sigma(j)=q$ for some $j\neq i$. Thus, for all permutations, this product is zero and
%	\begin{equation}
%	\bra{\psi_1}\sum_{i=1}^n \hat h_i \ket{\psi_2} = 0
%	\end{equation}
%	
%	The matrix element of the two electron operator reduces the same way as before to Equation \eqref{eq:gexp}.
%%	\begin{gather}
%%	\begin{gathered}
%%	\bra{\psi_1}\frac{1}{2}\sum_{i=1}^n\sum_{j\neq i}^n \hat g_{ij}\ket{\psi_2} = \\ 
%%	\frac{1}{2}\frac{1}{n!} \sum_{\sigma \in S_n} \sum_{i \neq j}^n  \text{sign}(\sigma) \\ \times \left[ \bra{\phi_{\sigma(i)}\phi_{\sigma(j)}}\hat g_{ij} \ket{\phi_{\sigma(i)}\phi_{\sigma(j)}} - \right.\\ \left. \bra{\phi_{\sigma(i)}\phi_{\sigma(j)}}\hat g_{ij} \ket{\phi_{\sigma(j)}\phi_{\sigma(i)}} \right] \\
%%	\times \prod_{k \neq i,j}^n \bra{\phi_{\sigma(k)}}\ket{\phi_{\sigma^\prime(k)}}
%%	\end{gathered}
%%	\end{gather}
%	Again, $\sigma(k)$ must not be equal to $p$ or $q$ ($k$ or $l$) if the term is to be nonzero. This forces $\sigma(i) = p$ and $\sigma(j) = k$ or the reverse ($k$ and $l$ for $\ket{\psi_1}$).
%		\begin{gather}
%	\begin{gathered}
%	= \frac{1}{2}\frac{1}{n!} \sum_{i \neq j}^n \sum_{\sigma\in S_n} \text{sign}(\sigma) \times \\ \left[ \bra{\phi_{k}\phi_{l}}\hat g_{ij} \ket{\phi_{p}\phi_{q}} - \right.
%	\\ \left. \bra{\phi_{k}\phi_{l}}\hat g_{ij} \ket{\phi_{q}\phi_{p}} \right] \\
%	+ \left[ \bra{\phi_{l}\phi_{k}}\hat g_{ij} \ket{\phi_{q}\phi_{p}} - \right.\\ \left. \bra{\phi_{k}\phi_{l}}\hat g_{ij} \ket{\phi_{p}\phi_{q}} \right] 
%	\end{gathered}
%	\end{gather}
%	The first and third term are equal after switching the dummy variable of integration. Same applies to the second and fourth term. There are $n^2-n = n(n-1)$ terms in the sum over $i \neq j$ and there are $(n-2)!$ permutations of the other $n-2$ elements.
%	\begin{gather}
%	\begin{gathered}
%	= \frac{1}{2}\frac{1}{n!}2 n(n-1)(n-2)! \\ \times \left[ \bra{\phi_{k}\phi_{l}}\hat g_{ij} \ket{\phi_{p}\phi_{q}} - \right.
%	\left. \bra{\phi_{k}\phi_{l}}\hat g_{ij} \ket{\phi_{q}\phi_{p}} \right] \\
%	\end{gathered}\\
%	=(kp|lq) - (kq|lp)
%	\end{gather}
%	
%	And for the zero electron operator.
%	\begin{gather}
%	\bra{\psi_1} \hat V_{NN} \ket{\psi_2} = V_{NN}\bra{\psi_1}\ket{\psi_2} = 0
%	\end{gather}
%	
%	\item The matrix element of the one electron operator reduces to Equation \eqref{eq:hexp}. Again, if $\sigma(j)$ is $p, q,$ or $r$, the term in the sum is zero. Since the product loops over all but one element, it must be that $\sigma(j)$ takes on one of these values and so the sum totals to zero.
%		\begin{equation}
%	\bra{\psi_1}\sum_{i=1}^n \hat h_i \ket{\psi_2} = 0
%	\end{equation}
%	
%	The matrix element of the two electron operator reduces to Equation \eqref{eq:gexp}. If $\sigma(k)$ takes on $p$, $q$, or $r$, then the term in the sum is zero. However, since $k$ loops over all but two elements, it must take on at least one of these values in the repeated product. So every term in the sum must be zero.
%		\begin{equation}
%	\bra{\psi_1}\sum_{j>i}^n \hat g_{ij} \ket{\psi_2} = 0
%	\end{equation}
%	
%	The matrix element of the zero electron operator is
%		\begin{gather}
%	\bra{\psi_1} \hat V_{NN} \ket{\psi_2} = V_{NN}\bra{\psi_1}\ket{\psi_2} = 0
%	\end{gather}
%	
%\end{enumerate}




%% References
%%
%% Following citation commands can be used in the body text:
%% Usage of \cite is as follows:
%%   \cite{key}         ==>>  [#]
%%   \cite[chap. 2]{key} ==>> [#, chap. 2]
%%

%% References with bibTeX database:

\bibliographystyle{elsarticle-num}
\bibliography{references}

%% Authors are advised to submit their bibtex database files. They are
%% requested to list a bibtex style file in the manuscript if they do
%% not want to use elsarticle-num.bst.

%% References without bibTeX database:

% \begin{thebibliography}{00}

%% \bibitem must have the following form:
%%   \bibitem{key}...
%%

% \bibitem{}

% \end{thebibliography}


\end{document}

%%
%% End of file `elsarticle-template-num.tex'.
