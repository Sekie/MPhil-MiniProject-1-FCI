%%
%% Copyright 2007, 2008, 2009 Elsevier Ltd
%%
%% This file is part of the 'Elsarticle Bundle'.
%% ---------------------------------------------
%%
%% It may be distributed under the conditions of the LaTeX Project Public
%% License, either version 1.2 of this license or (at your option) any
%% later version.  The latest version of this license is in
%%    http://www.latex-project.org/lppl.txt
%% and version 1.2 or later is part of all distributions of LaTeX
%% version 1999/12/01 or later.
%%
%% This template has been modified by Philip Blakely for
%% local distribution to students on the MPhil for Scientific
%% Computing course run at the University of Cambridge.
%%

%% Template article for Elsevier's document class `elsarticle'
%% with numbered style bibliographic references
%% SP 2008/03/01
%%
%%
%%
%% $Id: elsarticle-template-num.tex 4 2009-10-24 08:22:58Z rishi $
%%
%%
\documentclass[final,3p,times,twocolumn]{elsarticle}

%% Use the option review to obtain double line spacing
%% \documentclass[preprint,review,12pt]{elsarticle}

%% Use the options 1p,twocolumn; 3p; 3p,twocolumn; 5p; or 5p,twocolumn
%% for a journal layout:
%% \documentclass[final,1p,times]{elsarticle}
%% \documentclass[final,1p,times,twocolumn]{elsarticle}
%% \documentclass[final,3p,times]{elsarticle}
%% \documentclass[final,3p,times,twocolumn]{elsarticle}
%% \documentclass[final,5p,times]{elsarticle}
%% \documentclass[final,5p,times,twocolumn]{elsarticle}

%% if you use PostScript figures in your article
%% use the graphics package for simple commands
%% \usepackage{graphics}
%% or use the graphicx package for more complicated commands
%% \usepackage{graphicx}
%% or use the epsfig package if you prefer to use the old commands
%% \usepackage{epsfig}

%% The amssymb package provides various useful mathematical symbols
\usepackage{amssymb}
%% The amsthm package provides extended theorem environments
%% \usepackage{amsthm}

%% The lineno packages adds line numbers. Start line numbering with
%% \begin{linenumbers}, end it with \end{linenumbers}. Or switch it on
%% for the whole article with \linenumbers after \end{frontmatter}.
%% \usepackage{lineno}

%% natbib.sty is loaded by default. However, natbib options can be
%% provided with \biboptions{...} command. Following options are
%% valid:

%%   round  -  round parentheses are used (default)
%%   square -  square brackets are used   [option]
%%   curly  -  curly braces are used      {option}
%%   angle  -  angle brackets are used    <option>
%%   semicolon  -  multiple citations separated by semi-colon
%%   colon  - same as semicolon, an earlier confusion
%%   comma  -  separated by comma
%%   numbers-  selects numerical citations
%%   super  -  numerical citations as superscripts
%%   sort   -  sorts multiple citations according to order in ref. list
%%   sort&compress   -  like sort, but also compresses numerical citations
%%   compress - compresses without sorting
%%
%% \biboptions{comma,round}

% \biboptions{}

\usepackage{physics}
\usepackage{graphicx}


\journal{MPhil in Scientific Computing}

\begin{document}

\begin{frontmatter}

%% Title, authors and addresses

%% use the tnoteref command within \title for footnotes;
%% use the tnotetext command for the associated footnote;
%% use the fnref command within \author or \address for footnotes;
%% use the fntext command for the associated footnote;
%% use the corref command within \author for corresponding author footnotes;
%% use the cortext command for the associated footnote;
%% use the ead command for the email address,
%% and the form \ead[url] for the home page:
%%
%% \title{Title\tnoteref{label1}}
%% \tnotetext[label1]{}
%% \author{Name\corref{cor1}\fnref{label2}}
%% \ead{email address}
%% \ead[url]{home page}
%% \fntext[label2]{}
%% \cortext[cor1]{}
%% \address{Address\fnref{label3}}
%% \fntext[label3]{}

\title{Full Configuration Interaction Calculations on BH and H$_2$O}

%% use optional labels to link authors explicitly to addresses:
%% \author[label1,label2]{<author name>}
%% \address[label1]{<address>}
%% \address[label2]{<address>}

\author{Henry Tran}

\address{Department of Chemistry, Lensfield Road, Cambridge, UK,
CB2 1EW}

\begin{abstract}
ABSTRACT
\end{abstract}

\end{frontmatter}

%%
%% Start line numbering here if you want
%%
% \linenumbers

%% main text
\section{Introduction}
The method of configuration interaction (CI) is ubiquitous in computational chemistry with a rich and varied history. Expanding upon the Hartree-Fock (HF) equations,\cite{hartree,fock} CI introduces correlation through coupling the single HF configuration with other configurations.

\section{Mathematical formulae}
\label{sect:Formulae}
Complex formulae are easy to produce within \LaTeX{}. For example, we
can use inline equations to define $f(x) = a_3x^3 + a_2x^2 + a_1x +
a_0$, and then provide more complex equations such as
\begin{equation}
\int_0^3 f(x) \mathrm{d}x = \left[\frac{a_3}{4}x^4 + \frac{a_2}{3}x^3
  + \frac{a_1}{2}x^2 + a_0x\right]_0^3
\label{eqn:Taylor}
\end{equation}
If we have some derivation that should belong elsewhere, we can put it
in an appendix such as \ref{app:quad}. We can also refer to equations
from the main text such as the Taylor expansion \ref{eqn:Taylor}.
\section{Pictures}
\label{sect:Pictures}
In this section we demonstrate the inclusion of figures. For example,
Figure~\ref{fig:ENO1} demonstrates ENO as used to solve a linear advection
of a top-hat function.
\begin{figure}
\centering
%\includegraphics[width=3in]{ENOTest3b.png}
\caption{Demonstration of ENO as used to solver linear-advection of a
  top-hat function.}
\label{fig:ENO1}
\end{figure}
\section{Conclusions}
\label{sect:Concl}
In which we conclude that LaTeX (or \LaTeX) is very useful for
generating scientific papers as demonstrated above.

\section*{Acknowledgements}
Here I acknowledge the assistance of my supervisor, my industrial sponsor,
and the effects of caffeine on my ability to produce this report on time.

%% The Appendices part is started with the command \appendix;
%% appendix sections are then done as normal sections
\appendix

\section{On the Derivation of the Quadratic Formula}
\label{app:quad}
The derivation of the quadratic formula is something that would not
fit well within a paper as it would interrupt the flow of the argument
therein. However, for those students who need a refresher on how the
quadratic formula is derived, we give full details here:\par
Assume that we have
\begin{equation}
p(x) = ax^2 + bx + c
\end{equation}
and so on. The actual algebra is left as an exercise for the reader.

%% References
%%
%% Following citation commands can be used in the body text:
%% Usage of \cite is as follows:
%%   \cite{key}         ==>>  [#]
%%   \cite[chap. 2]{key} ==>> [#, chap. 2]
%%

%% References with bibTeX database:

\bibliographystyle{elsarticle-num}
\bibliography{references.bib}

%% Authors are advised to submit their bibtex database files. They are
%% requested to list a bibtex style file in the manuscript if they do
%% not want to use elsarticle-num.bst.

%% References without bibTeX database:

% \begin{thebibliography}{00}

%% \bibitem must have the following form:
%%   \bibitem{key}...
%%

% \bibitem{}

% \end{thebibliography}


\end{document}

%%
%% End of file `elsarticle-template-num.tex'.
